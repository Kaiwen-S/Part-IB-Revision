\documentclass{article}
\usepackage{graphicx} % Required for inserting images
\usepackage[utf8]{inputenc}
\usepackage{amsmath,amsfonts,amssymb,amsthm}
\usepackage{enumerate,bbm}
\usepackage{tikz,graphicx,color,mathrsfs,color,hyperref}
\usepackage{caption,float}
\usepackage[a4paper,margin=1in,footskip=0.25in]{geometry}

\usepackage{listings}
\usepackage{xcolor}

\usepackage{tabularx,capt-of}

\usepackage{blindtext}
%Image-related packages
\usepackage{graphicx}
\usepackage{subcaption}
\usepackage[export]{adjustbox}
\usepackage{lipsum}

%hyperref setup
\hypersetup{
    colorlinks=true,
    linkcolor=blue,
    filecolor=magenta,      
    urlcolor=cyan,
    pdftitle={Overleaf Example},
    pdfpagemode=FullScreen,
    }

%New colors defined below
\definecolor{codegreen}{rgb}{0,0.6,0}
\definecolor{codegray}{rgb}{0.5,0.5,0.5}
\definecolor{codepurple}{rgb}{0.58,0,0.82}
\definecolor{backcolour}{rgb}{0.95,0.95,0.92}

%Code listing style named "mystyle"
\lstdefinestyle{mystyle}{
  backgroundcolor=\color{backcolour}, commentstyle=\color{codegreen},
  keywordstyle=\color{magenta},
  numberstyle=\tiny\color{codegray},
  stringstyle=\color{codepurple},
  basicstyle=\ttfamily\footnotesize,
  breakatwhitespace=false,         
  breaklines=true,                 
  captionpos=b,                    
  keepspaces=true,                 
  numbers=left,                    
  numbersep=5pt,                  
  showspaces=false,                
  showstringspaces=false,
  showtabs=false,                  
  tabsize=2
}

%"mystyle" code listing set
\lstset{style=mystyle}

\theoremstyle{definition}
\newtheorem{defn}{Definition}[section]
\theoremstyle{remark}
\newtheorem{rem}{Remark}
\newtheorem{remS}[section]{defn}
\newtheorem{lem}[defn]{Lemma}
\theoremstyle{plain}
\newtheorem{thm}[defn]{Theorem}
\newtheorem{prop}[defn]{Proposition}
\newtheorem{fact}[defn]{Fact}
\newtheorem{crly}[defn]{Corollary}
\newtheorem{conj}[defn]{Conjecture}
\newtheorem{example}{Example}
\theoremstyle{definition}
%\newtheorem*{programming*}{Programming Task}

%\newtheorem{innercustomgeneric}{\customgenericname}
%\providecommand{\customgenericname}{}
%\newcommand{\newcustomtheorem}[2]{%
%  \newenvironment{#1}[1]
%  {%
%   \renewcommand\customgenericname{#2}%
%   \renewcommand\theinnercustomgeneric{##1}%
%   \innercustomgeneric
%  }
%  {\endinnercustomgeneric}
%}

%\newcustomtheorem{question}{Question}
%\newcustomtheorem{programming}{Programming Task}

\newcommand{\NN}{\mathbb{N}}
\newcommand{\ZZ}{\mathbb{Z}}
\newcommand{\QQ}{\mathbb{Q}}
\newcommand{\RR}{\mathbb{R}}
\newcommand{\CC}{\mathbb{C}}
\newcommand{\PP}{\mathbb{P}}

\newcommand{\calD}{\mathcal{D}}

\title{Methods}
\author{ }
\date{ }

\begin{document}
\maketitle
\section{Fourier stuff}
Fourier series for $L$-periodic functions
\begin{align*}
    f(x)\sim \sum_{n=-\infty}^\infty c_n e^{i2\pi n x/L}=\dfrac{1}{2}a_0+\sum_{n=1}^\infty a_n\cos(2\pi nx/L)+\sum_{n=0}^\infty b_n\sin(2\pi nx/L)
\end{align*}
where 
\begin{align*}
    &c_n=\dfrac{1}{L}\int_0^L f(x)e^{i2\pi nx/L}dx\\
    &a_n=\dfrac{2}{L}\int_0^L f(x)\cos(2\pi nx/L)dx\\
    &b_n=\dfrac{2}{L}\int_0^L f(x)\sin(2\pi nx/L)dx
\end{align*}
Fourier sin/cos series for odd/even extension. $f:[0,L)\to\CC$, then have odd/even periodic extension.
\begin{align*}
    &f_{\text{odd}}(x)\sim \sum_{n=0}^\infty b_n\sin(n\pi x/L)\\
    &f_{\text{even}}(x)\sim\dfrac{1}{2}a_0+\sum_{n=1}^\infty a_n\cos(n\pi x/L)
\end{align*}
where
\[a_n=\dfrac{2}{L}\int_0^{L}f(x)\cos(n\pi x/L)dx,\ b_n=\dfrac{2}{L}\int_0^L f(x)\sin(n\pi x/L)\]
Fourier transform of $f:\RR\to\CC$
\[\hat f(\lambda)=\int_{-\infty}^\infty f(x)e^{-i\lambda x}dx\]
(Easily generalizes to multivariable case.)
Inverse transform
\[f(x)=\dfrac{1}{2\pi}\int_{-\infty}^\infty \hat f(\lambda)e^{i\lambda x}d\lambda\]
\begin{thm}[Parseval]
    Let $f\sim\sum \hat f_ne^{i2\pi nx/L}$ $g\sim \sum\hat g_ne^{i2\pi nx/L}$, then
    \[\dfrac{1}{L}\langle f,g\rangle=\dfrac{1}{L}\int_0^Lf(x)\overline{g(x)}dx=\sum \hat f_n\overline{\hat g_n}\]
    In terms of real fourier series coeff, 
    \[=\dfrac{1}{2}a_0^2+\sum_{n=1}^\infty (a_n^2+b_n^2)\]
    For Fourier transform, have
    \[\int_{-\infty}^\infty f(x)g(x)dx=\dfrac{1}{2\pi}\int_{-\infty}^\infty \hat f(\lambda) \hat g(\lambda)d\lambda\]
\end{thm}
\begin{prop}[Properties of Fourier transform]
    \begin{itemize}
        \item Fourier transform of $f^{(k)}$ = $(i\lambda)^k$ times $\hat f$. Fourier transform of $x^k f(x)$ = $i^k$ times the kth derivative of $\hat f$.
        \item Fourier transform of $f(x-a)$ = $e^{-i\lambda a}$. Fourier transform of $e^{iax}f(x)$ = $\hat f(x+a)$.
        \item Fourier transform of convolution is the product of Fourier transforms of each individual function.
    \end{itemize}
\end{prop}

\section{Sturm-Liouville stuff}
\begin{prop}
    Self-adjointness.
\end{prop}
\begin{proof}
\end{proof}
Reduction to Sturm-Liouville form
\[\mathcal{L}=\dfrac{1}{w}\left[\dfrac{d}{dx}\left(-p\dfrac{d}{dx}\right)+q\right]\]
Given
\[\alpha(x)\dfrac{d^2}{dx^2}+\beta(x)\dfrac{d}{dx}+\gamma(x)y\]
Then, can reduce to Sturm-Liouville form by first setting $I(x)=\exp(\int \beta/\alpha)$ (integrating factor)
\begin{align*}
    p=I(x),\ q=-\dfrac{\gamma(x)I(x)}{\alpha(x)}, \ w(x)=\dfrac{I(x)}{\alpha(x)}
\end{align*}
$w$ is called the weight function. Eigenfunctions of $\mathcal{L}$ are orthogonal under the inner product $\langle f,g\rangle_w=\int_a^b f\bar{g}wdx$ (complex conjugate can be omitted if working over $\RR$)
\subsection{Important examples}
\begin{example}[Legendre polynomials]
    
\end{example}
\begin{example}[Hermite polynomials]
    
\end{example}
\begin{example}[Bessel's functions]
    
\end{example}
\section{Green's function}
Let $L=\alpha(x) d^2/dx^2+\beta(x)d/dx+\gamma(x)$ be a second order differential operator, $f$ some functions on $[a,b]$
\begin{align*}
    \begin{cases}
        Ly=f\\
        y(a)=0\\
        y(b)=0
    \end{cases}
\end{align*}
Green's function is given by
\begin{align*}
    G(x;\xi)=\dfrac{1}{\alpha(\xi)W(y_1,y_2)(\xi)}\begin{cases}
        y_1(x)y_2(\xi) & a<x<\xi\\
        y_1(\xi)y_2(x) & \xi<x<b
    \end{cases}
\end{align*}
where $y_1$ and $y_2$ are linearly indep solns of $Ly=0$ s.t. $y_1(a)=0$ and $y_2(b)=0$. Note that the jump discontinuity of $G'(x;\xi)$ is given by $1/(\alpha(\xi))$.
Then the solution to the inhomogeneous problem is given by
\begin{align*}
    y(x)=y_2(x)\int_a^x\dfrac{y_1(\xi)f(\xi)}{\alpha(\xi)W(y_1,y_2)(\xi)}d\xi+ y_1(x)\int_x^b\dfrac{y_2(\xi)f(\xi)}{\alpha(\xi)W(y_1,y_2)(\xi)}d\xi
\end{align*}
Similar situation applies to initial value problem (jump discontinuity given similarly). General solution for IVP is
\begin{align*}
    y(t)=\int_0^\infty G(t;\tau)f(\tau)d\tau=\int_0^tG(t;\tau)f(\tau)d\tau
\end{align*}
\section{PDEs}
\subsection{Methods of characteristics}
Consider quasi-linear PDEs of the form
\[a(x,y)\dfrac{\partial u}{\partial x}+b(x,y)\dfrac{\partial u}{\partial y}=c(x,y,u)\]
Characteristic curves are given by solving
\[\begin{cases}
    \dot x=a(x,y)\\
    \dot y=b(x,y)
\end{cases}\]
Initial condition $(x(0),y(0))$ on a curve $C$ (depending on the initial condition of the PDE problem).
Need to make sure that the vector field $(a,b)$ is nowhere parallel to the tangent of $C$ (non-characteristic condition).
\subsection{Canonical forms}

\subsection{Laplace equation}
\subsubsection{Separation of variables}
In spherical polar (axis-symmetric case), need to substitute $x=\cos(\theta)$ and use Legendre polynomials.
\subsubsection{Fourier transform}
Free space green's functions
\begin{align*}
    G({\bf x};{\bf y})=\begin{cases}
        \dfrac{1}{2\pi}\log\vert{\bf x}-{\bf y}\vert & n=2\\
        \dfrac{1}{(n-2)\operatorname{Area}(S^{n-1})}\dfrac{1}{\vert{\bf x}-{\bf y}\vert^{n-2}} & n\ge 3
    \end{cases}
\end{align*}
Can construct Dirichlet green's function $\mathcal{G}({\bf x};{\bf y})$ by method of images.
\subsection{Wave equation}
\[\phi_{tt}-c^2\nabla^2\phi=0\]
\subsubsection{Separation of variables}
\subsubsection{Fourier transform}
Important kernel:
\[\hat\Phi_t(\vec\lambda)=\dfrac{\sin(c|\vec\lambda|t)}{c|\vec\lambda|}\]
For $n=1$,
\[\Phi_t(x)=\dfrac{1}{2c}H(x+ct)-\dfrac{1}{2c}H(x-ct)\]
For $n=3$, (compute inverse transform using spherical polar)
\[\Phi_t(\vec x)=\dfrac{1}{4\pi c|\vec x|}\delta(|\vec x|-ct)-\dfrac{1}{4\pi c|\vec x|}\delta(|\vec x|+ct)\]
Take Fourier transform with respect to $x$ or $\vec x$ if in higher dimensions. Solve the homogeneous problem with non-trivial boundary condition. Then solve the inhomogeneous problem with Dirichlet boundary condition using Green's function, which is the same as $G(t;s)=\hat\Phi_{t-s}(\lambda)$. Then apply convolution theorem.

On half line, simply consider the odd extension to the whole real line and apply the above.
\subsection{Heat equation}
\[u_t-\kappa\nabla^2 u=0\]
\subsubsection{Separation of variables}
\subsubsection{Fourier transform}
Heat kernel
\[K_t(\vec x)=\dfrac{1}{(4\pi\kappa t)^{n/2}}\exp\left(-\dfrac{|\vec x|^2}{4\kappa t}\right)\]
and its Fourier transform (with respect to $\vec x$)
\[\hat K_t(\vec\lambda)=\exp\left(-\kappa t|\vec\lambda|^2\right)\]
Take Fourier transform wrt $\vec x$. Solve homogeneous problem with non-trivial boundary condition first. Then solve inhomogeneous problem with trivial boundary condition (first order system so use integrating factor). Superpose.

\end{document}