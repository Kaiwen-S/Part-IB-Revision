\documentclass{article}
\usepackage{graphicx} % Required for inserting images
\usepackage[utf8]{inputenc}
\usepackage{amsmath,amsfonts,amssymb,amsthm}
\usepackage{enumerate,bbm}
\usepackage{tikz,graphicx,color,mathrsfs,color,hyperref}
\usepackage{caption,float}
\usepackage[a4paper,margin=1in,footskip=0.25in]{geometry}

\usepackage{listings}
\usepackage{xcolor}

\usepackage{tabularx,capt-of}

\usepackage{blindtext}
%Image-related packages
\usepackage{graphicx}
\usepackage{subcaption}
\usepackage[export]{adjustbox}
\usepackage{lipsum}

%hyperref setup
\hypersetup{
    colorlinks=true,
    linkcolor=blue,
    filecolor=magenta,      
    urlcolor=cyan,
    pdftitle={Overleaf Example},
    pdfpagemode=FullScreen,
    }

%New colors defined below
\definecolor{codegreen}{rgb}{0,0.6,0}
\definecolor{codegray}{rgb}{0.5,0.5,0.5}
\definecolor{codepurple}{rgb}{0.58,0,0.82}
\definecolor{backcolour}{rgb}{0.95,0.95,0.92}

%Code listing style named "mystyle"
\lstdefinestyle{mystyle}{
  backgroundcolor=\color{backcolour}, commentstyle=\color{codegreen},
  keywordstyle=\color{magenta},
  numberstyle=\tiny\color{codegray},
  stringstyle=\color{codepurple},
  basicstyle=\ttfamily\footnotesize,
  breakatwhitespace=false,         
  breaklines=true,                 
  captionpos=b,                    
  keepspaces=true,                 
  numbers=left,                    
  numbersep=5pt,                  
  showspaces=false,                
  showstringspaces=false,
  showtabs=false,                  
  tabsize=2
}

%"mystyle" code listing set
\lstset{style=mystyle}

\theoremstyle{definition}
\newtheorem{defn}{Definition}[section]
\theoremstyle{remark}
\newtheorem{rem}{Remark}
\newtheorem{remS}[section]{defn}
\newtheorem{lem}[defn]{Lemma}
\theoremstyle{plain}
\newtheorem{thm}[defn]{Theorem}
\newtheorem{prop}[defn]{Proposition}
\newtheorem{fact}[defn]{Fact}
\newtheorem{crly}[defn]{Corollary}
\newtheorem{conj}[defn]{Conjecture}
\newtheorem{example}{Example}
\theoremstyle{definition}
%\newtheorem*{programming*}{Programming Task}

%\newtheorem{innercustomgeneric}{\customgenericname}
%\providecommand{\customgenericname}{}
%\newcommand{\newcustomtheorem}[2]{%
%  \newenvironment{#1}[1]
%  {%
%   \renewcommand\customgenericname{#2}%
%   \renewcommand\theinnercustomgeneric{##1}%
%   \innercustomgeneric
%  }
%  {\endinnercustomgeneric}
%}

%\newcustomtheorem{question}{Question}
%\newcustomtheorem{programming}{Programming Task}

\newcommand{\NN}{\mathbb{N}}
\newcommand{\ZZ}{\mathbb{Z}}
\newcommand{\QQ}{\mathbb{Q}}
\newcommand{\RR}{\mathbb{R}}
\newcommand{\CC}{\mathbb{C}}
\newcommand{\PP}{\mathbb{P}}

\newcommand{\calD}{\mathcal{D}}

\title{EM}
\author{ }
\date{ }

\begin{document}
\maketitle
\section{Important equation/identity/formula}
An important relation
\[\dfrac{1}{c^2}=\mu_0\epsilon_0\]
Charge conservation
\[\dfrac{\partial\rho}{\partial t}+\nabla\cdot\vec J=0\]
Maxwell's equations
\begin{align*}
    \begin{cases}
    \nabla\cdot\vec E=\dfrac{\rho}{\epsilon_0}\\
    \nabla\cdot \vec B=0\\
    \nabla\times \vec E=-\dfrac{\partial\vec B}{\partial t}\\
    \nabla\times \vec B=\mu_0 \vec J+\dfrac{1}{c^2}\dfrac{\partial \vec E}{\partial t}
    \end{cases}
\end{align*}
Lorentz force law (relativistic)
\[\dfrac{d}{dt}(m\gamma \vec v)=q[\vec E(t,\vec x)+\vec v\times \vec B(t,\vec x)]\]
Ohm's law (relating current density and electric field)
\[\vec J=\sigma\vec E\]
For moving conductors, Ohm's law becomes
\[\vec J=\sigma(\vec E+\vec v\times\vec B)\]
Poynting vector
\[\vec S=\dfrac{\vec E\times\vec B}{\mu_0}\]
Energy density of electromagnetic field
\[w=\dfrac{\epsilon_0}{2}\vec E^2+\dfrac{1}{2\mu_0}\vec B^2\]
Resistance (of a thin wire)
\[R=\dfrac{l}{\sigma A}\]

\subsection{Electric field (static)}
Electric potential $\vec E=-\nabla\phi$ (solve Poisson equation $\nabla^2\phi=-\rho/\epsilon_0$ using Green's function)
\[\phi(\vec x)=-\int_{\RR^3}\dfrac{\rho(\vec x')}{4\pi\epsilon_0|\vec x-\vec x'|}d^3\vec x'\]
Electric field is
\[\vec E(\vec x)=\int_{\RR^3}\dfrac{\rho(\vec x')(\vec x-\vec x')}{4\pi\epsilon_0|\vec x-\vec x'|^3}d^3\vec x'\]
Energy density of electric field
\[w=\dfrac{\epsilon_0}{2}\vec E^2\]
Gauss's Law
\[-\int_S\nabla\phi\cdot d\vec S=\int_S \vec E\cdot d\vec S=\dfrac{Q[V]}{\epsilon_0}\]
Surface charge and discontinuity in normal component
\[\vec E_+-\vec E_-=\dfrac{\sigma}{\epsilon_0}\vec n\]
The tangential component is continuous.
\subsection{Magnetic field (static)}
Vector potential $\vec B=\nabla\times\vec A$. (Solve Poisson equation $\nabla^2\vec A=\mu_0\vec J$)
\[\vec A(\vec x)=\int_{\RR^3}\dfrac{\mu_0\vec J(\vec x')}{4\pi|\vec x-\vec x'|}d^3\vec x'\]
Magnetic field is (expand by suffix notation)
\[\vec B(\vec x)=\int_{\RR^3}\dfrac{\mu_0 \vec J(\vec x')\times(\vec x-\vec x')}{4\pi|\vec x-\vec x'|^3}d^3\vec x'\]
Ampere's law
\[\int_C\vec B\cdot d\vec x=\mu_0 I[S]\]
Energy density
\[w=\dfrac{1}{2\mu_0}\vec B^2\]
Biot-Savart law (magnetic field produced by thin wire on curve $C$ with constant current $I$)
\[\vec B(\vec x)=\dfrac{\mu_0I}{2\pi}\int_{C}\dfrac{d\vec x'\times (\vec x-\vec x')}{|\vec x-\vec x'|^3}\]
Surface current and discontinuity
\[\vec B_+-\vec B_-=\mu_0\vec K\times \vec n\]

\subsection{Time dependence}
Faraday's law (integrate M3 over a fixed surface and apply Stokes)
\[\mathcal{E}(t)=-\dfrac{d\mathcal{F}}{dt}\]
Faraday's law for time-dependent region
\[-\dfrac{d\mathcal F}{dt}=\int_{C(t)}(\vec E(t,\vec x)+\vec{v}\times \vec B(t,\vec x))d\vec x=\mathcal E(t)\]

\end{document}