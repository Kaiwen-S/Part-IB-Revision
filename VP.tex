\documentclass{article}
\usepackage{graphicx} % Required for inserting images
\usepackage[utf8]{inputenc}
\usepackage{amsmath,amsfonts,amssymb,amsthm}
\usepackage{enumerate,bbm}
\usepackage{tikz,graphicx,color,mathrsfs,color,hyperref}
\usepackage{caption,float}
\usepackage[a4paper,margin=1in,footskip=0.25in]{geometry}

\usepackage{listings}
\usepackage{xcolor}

\usepackage{tabularx,capt-of}

\usepackage{blindtext}
%Image-related packages
\usepackage{graphicx}
\usepackage{subcaption}
\usepackage[export]{adjustbox}
\usepackage{lipsum}

%hyperref setup
\hypersetup{
    colorlinks=true,
    linkcolor=blue,
    filecolor=magenta,      
    urlcolor=cyan,
    pdftitle={Overleaf Example},
    pdfpagemode=FullScreen,
    }

%New colors defined below
\definecolor{codegreen}{rgb}{0,0.6,0}
\definecolor{codegray}{rgb}{0.5,0.5,0.5}
\definecolor{codepurple}{rgb}{0.58,0,0.82}
\definecolor{backcolour}{rgb}{0.95,0.95,0.92}

%Code listing style named "mystyle"
\lstdefinestyle{mystyle}{
  backgroundcolor=\color{backcolour}, commentstyle=\color{codegreen},
  keywordstyle=\color{magenta},
  numberstyle=\tiny\color{codegray},
  stringstyle=\color{codepurple},
  basicstyle=\ttfamily\footnotesize,
  breakatwhitespace=false,         
  breaklines=true,                 
  captionpos=b,                    
  keepspaces=true,                 
  numbers=left,                    
  numbersep=5pt,                  
  showspaces=false,                
  showstringspaces=false,
  showtabs=false,                  
  tabsize=2
}

%"mystyle" code listing set
\lstset{style=mystyle}

\theoremstyle{definition}
\newtheorem{defn}{Definition}[section]
\theoremstyle{remark}
\newtheorem{rem}{Remark}
\newtheorem{remS}[section]{defn}
\newtheorem{lem}[defn]{Lemma}
\theoremstyle{plain}
\newtheorem{thm}[defn]{Theorem}
\newtheorem{prop}[defn]{Proposition}
\newtheorem{fact}[defn]{Fact}
\newtheorem{crly}[defn]{Corollary}
\newtheorem{conj}[defn]{Conjecture}
%\newtheorem*{programming*}{Programming Task}

%\newtheorem{innercustomgeneric}{\customgenericname}
%\providecommand{\customgenericname}{}
%\newcommand{\newcustomtheorem}[2]{%
%  \newenvironment{#1}[1]
%  {%
%   \renewcommand\customgenericname{#2}%
%   \renewcommand\theinnercustomgeneric{##1}%
%   \innercustomgeneric
%  }
%  {\endinnercustomgeneric}
%}

%\newcustomtheorem{question}{Question}
%\newcustomtheorem{programming}{Programming Task}

\newcommand{\NN}{\mathbb{N}}
\newcommand{\ZZ}{\mathbb{Z}}
\newcommand{\QQ}{\mathbb{Q}}
\newcommand{\RR}{\mathbb{R}}
\newcommand{\CC}{\mathbb{C}}
\newcommand{\PP}{\mathbb{P}}

\newcommand{\calD}{\mathcal{D}}

\title{VP}
\author{ }
\date{ }

\begin{document}

\maketitle

\section{EL equations}
Functional:
\begin{align}
    F[y]=\int_a^b f(x,y,y')dx
\end{align}
Euler-Lagrange equation (basic):
\begin{align*}
    \dfrac{\partial f}{\partial y}-\dfrac{d}{dx}\dfrac{\partial f}{\partial y'}=0
\end{align*}
Euler-Lagrange equation (first integral eliminating $x$):
\begin{align*}
    f-y'\dfrac{\partial f}{\partial y'}=\text{Constant}
\end{align*}
Euler-Lagrange equation (first integral eliminating $y$):
\begin{align*}
    \dfrac{\partial f}{\partial y'}=\text{Constant}
\end{align*}
Euler-Lagrange equation with constraint $G[y]=0$ (Lagrange multiplier): \textbf{Apply EL equation to $F[y]-\lambda G[y]$.}\\
Euler-Lagrange equation (multiple dependent variables), i.e., $\int_a^b f(x,y_1,\ldots,y_n, y_1',\ldots,y_n')dx$: 
\begin{align*}
    \dfrac{\partial f}{\partial y_i}-\dfrac{d}{dx}\dfrac{\partial f}{\partial y_i'}=0
\end{align*}
for each $i$. (So this is a system of DEs)\\
First integral eliminating $x$:
\begin{align*}
    f-\sum_iy_i'\dfrac{\partial f}{\partial y_i'}=\text{Constant}
\end{align*}
Euler-Lagrange equation (multiple independent variables), e.g., $\int_V f(x_1,\ldots,x_n,y,y_{x_1},\ldots,y_{x_n})dV$:\\
(summation convention applies)
\begin{align*}
    \dfrac{\partial f}{\partial y}-\dfrac{\partial f}{\partial x_i}\dfrac{\partial f}{\partial y_{x_i}}=0
\end{align*}
Euler-Lagrange equation (higher derivatives), e.g., $\int_a^b f(x,y,y^{(1)},\ldots,y^{(n)})dx$:
\begin{align*}
    \dfrac{\partial f}{\partial y}-\sum_{i=1}^n(-1)^i\dfrac{d^i}{dx^i}\dfrac{\partial f}{\partial y^{(i)}}=0
\end{align*}


\section{Principles}
Fermat's principle:\\
Least action principle:

\section{Legendre Transform}
Let $f:S\to \RR$, $S\subseteq\RR^n$. The Legendre transform is given by
\begin{align*}
    f^\ast({\bf p})=\sup_{{\bf x}\in S}({\bf p}\cdot {\bf x}-f({\bf x}))
\end{align*}
Can take Legendre transform w.r.t one variable.

\section{Second Variation}
Second variation (basic form): $y\mapsto y+\epsilon\eta$
\begin{align*}
    \delta^2F[y]=\dfrac{1}{2}\int_a^b\left(\eta^2\dfrac{\partial^2 f}{\partial y^2}+2\eta\eta'\dfrac{\partial^2 f}{\partial y\partial y'}+\eta'^2\dfrac{\partial^2 f}{\partial y'^2}\right)dx
\end{align*}
Second variation (integrate the middle term by part):
\begin{align*}
    \delta^2F[y]=\dfrac{1}{2}\int_a^b(Q\eta^2+P(\eta')^2)dx
\end{align*}
\begin{align*}
    Q=\dfrac{\partial^2 f}{\partial y^2}-\dfrac{d}{dx}\dfrac{\partial^2 f}{\partial y\partial y'},\ \ \ \ P=\dfrac{\partial^2 f}{\partial y'^2}
\end{align*}
\begin{thm}[Jacobi accessory condition]
    If there exists a nowhere vanishing solution to \[-(Pu')'+Qu=0\]
    then $\delta^2F[y]>0$.
\end{thm}

\section{Important tricks}
Prove global optimum:
\begin{itemize}
    \item Complete the square (integrand or multivariate functions)
    \item Consider $F[y+\eta]-F[y]$. Do not assume $\eta$ is a small variation.
\end{itemize}
Integration techniques:
\begin{enumerate}
    \item IBP and sub.
    \item Product rule in some way.
    \item Add 0.
\end{enumerate}

\end{document}
